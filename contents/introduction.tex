%% introductioin.tex
%% Copyright 2022 skyleaworlder
%
% This work may be distributed and/or modified under the
% conditions of the LaTeX Project Public License, either version 1.3
% of this license or (at your option) any later version.
% The latest version of this license is in
%   http://www.latex-project.org/lppl.txt
% and version 1.3 or later is part of all distributions of LaTeX
% version 2003/12/01 or later.
%
% This work has the LPPL maintenance status "maintained".
%
% This Current Maintainer of this work is skyleaworlder.
%
% This work consists of all the *.tex and *.sty files in
%   https://github.com/TJ-CSCCG/Tongji-Beamer
\section{引言}
    \begin{frame}
    % “无序列表” 与 “有序列表” 使用
    \frametitle{引言}
        \footnotesize
        \begin{block}{研究目的}
            \begin{itemize}
                \item 软件维护是指软件系统交付后为更正软件缺陷或添加新功能的修改软件的活动。随着现代软件规模越来越大,软件维护也变得极其困难,其中代码变更分析就是关键难题之一。
                \item 代码变更分析聚焦于分析变更代码对存量代码的影响。当代码变更后,仅对受到影响的代码进行维护,避免对全量代码维护,进而降低软件维护的成本。
                \item AOSP Frameworks 作为 AOSP 的核心组成部分,代码量庞大、更迭速度快、重要性高,进行变更分析的意义重大。
            \end{itemize}
        \end{block}

        \begin{block}{研究内容}
            \begin{enumerate}
                \item 解耦致使 AOSP Frameworks 分析困难的问题。
                \item 在此基础上,对 AOSP Frameworks 进行模块间以及模块内的依赖分析。
                \item 开发可通过代码变更,分析 AOSP Framework 所受影响的系统。
            \end{enumerate}
        \end{block}
    \end{frame}

